%%
% Copyright (c) 2017 - 2025, Pascal Wagler;
% Copyright (c) 2014 - 2025, John MacFarlane
%
% All rights reserved.
%
% Redistribution and use in source and binary forms, with or without
% modification, are permitted provided that the following conditions
% are met:
%
% - Redistributions of source code must retain the above copyright
% notice, this list of conditions and the following disclaimer.
%
% - Redistributions in binary form must reproduce the above copyright
% notice, this list of conditions and the following disclaimer in the
% documentation and/or other materials provided with the distribution.
%
% - Neither the name of John MacFarlane nor the names of other
% contributors may be used to endorse or promote products derived
% from this software without specific prior written permission.
%
% THIS SOFTWARE IS PROVIDED BY THE COPYRIGHT HOLDERS AND CONTRIBUTORS
% "AS IS" AND ANY EXPRESS OR IMPLIED WARRANTIES, INCLUDING, BUT NOT
% LIMITED TO, THE IMPLIED WARRANTIES OF MERCHANTABILITY AND FITNESS
% FOR A PARTICULAR PURPOSE ARE DISCLAIMED. IN NO EVENT SHALL THE
% COPYRIGHT OWNER OR CONTRIBUTORS BE LIABLE FOR ANY DIRECT, INDIRECT,
% INCIDENTAL, SPECIAL, EXEMPLARY, OR CONSEQUENTIAL DAMAGES (INCLUDING,
% BUT NOT LIMITED TO, PROCUREMENT OF SUBSTITUTE GOODS OR SERVICES;
% LOSS OF USE, DATA, OR PROFITS; OR BUSINESS INTERRUPTION) HOWEVER
% CAUSED AND ON ANY THEORY OF LIABILITY, WHETHER IN CONTRACT, STRICT
% LIABILITY, OR TORT (INCLUDING NEGLIGENCE OR OTHERWISE) ARISING IN
% ANY WAY OUT OF THE USE OF THIS SOFTWARE, EVEN IF ADVISED OF THE
% POSSIBILITY OF SUCH DAMAGE.
%%

%%
% This is the Eisvogel pandoc LaTeX template.
%
% For usage information and examples visit the official GitHub page:
% https://github.com/Wandmalfarbe/pandoc-latex-template
%%
% Options for packages loaded elsewhere
\PassOptionsToPackage{unicode}{hyperref}
\PassOptionsToPackage{hyphens}{url}
\PassOptionsToPackage{dvipsnames,svgnames,x11names,table}{xcolor}
\documentclass[
  a4paper,
  ,captions=tableheading
]{scrartcl}
\usepackage{xcolor}
\usepackage[left=1cm,right=1cm,top=2.6cm,bottom=3.25cm]{geometry}
\usepackage{amsmath,amssymb}


% add backlinks to footnote references, cf. https://tex.stackexchange.com/questions/302266/make-footnote-clickable-both-ways
\usepackage{footnotebackref}
\setcounter{secnumdepth}{5}
\usepackage{iftex}
\ifPDFTeX
  \usepackage[T1]{fontenc}
  \usepackage[utf8]{inputenc}
  \usepackage{textcomp} % provide euro and other symbols
\else % if luatex or xetex
  \usepackage{unicode-math} % this also loads fontspec
  \defaultfontfeatures{Scale=MatchLowercase}
  \defaultfontfeatures[\rmfamily]{Ligatures=TeX,Scale=1}
\fi
\usepackage{lmodern}
\ifPDFTeX\else
  % xetex/luatex font selection
\fi
% Use upquote if available, for straight quotes in verbatim environments
\IfFileExists{upquote.sty}{\usepackage{upquote}}{}
\IfFileExists{microtype.sty}{% use microtype if available
  \usepackage[]{microtype}
  \UseMicrotypeSet[protrusion]{basicmath} % disable protrusion for tt fonts
}{}

% Use setspace anyway because we change the default line spacing.
% The spacing is changed early to affect the titlepage and the TOC.
\usepackage{setspace}
\setstretch{1.2}
\makeatletter
\@ifundefined{KOMAClassName}{% if non-KOMA class
  \IfFileExists{parskip.sty}{%
    \usepackage{parskip}
  }{% else
    \setlength{\parindent}{0pt}
    \setlength{\parskip}{6pt plus 2pt minus 1pt}}
}{% if KOMA class
  \KOMAoptions{parskip=half}}
\makeatother
\usepackage{longtable,booktabs,array}
\usepackage{calc} % for calculating minipage widths
% Correct order of tables after \paragraph or \subparagraph
\usepackage{etoolbox}
\makeatletter
\patchcmd\longtable{\par}{\if@noskipsec\mbox{}\fi\par}{}{}
\makeatother
% Allow footnotes in longtable head/foot
\IfFileExists{footnotehyper.sty}{\usepackage{footnotehyper}}{\usepackage{footnote}}
\makesavenoteenv{longtable}
% definitions for citeproc citations
\NewDocumentCommand\citeproctext{}{}
\NewDocumentCommand\citeproc{mm}{%
  \begingroup\def\citeproctext{#2}\cite{#1}\endgroup}
\makeatletter
 % allow citations to break across lines
 \let\@cite@ofmt\@firstofone
 % avoid brackets around text for \cite:
 \def\@biblabel#1{}
 \def\@cite#1#2{{#1\if@tempswa , #2\fi}}
\makeatother
\newlength{\cslhangindent}
\setlength{\cslhangindent}{1.5em}
\newlength{\csllabelwidth}
\setlength{\csllabelwidth}{3em}
\newenvironment{CSLReferences}[2] % #1 hanging-indent, #2 entry-spacing
 {\begin{list}{}{%
  \setlength{\itemindent}{0pt}
  \setlength{\leftmargin}{0pt}
  \setlength{\parsep}{0pt}
  % turn on hanging indent if param 1 is 1
  \ifodd #1
   \setlength{\leftmargin}{\cslhangindent}
   \setlength{\itemindent}{-1\cslhangindent}
  \fi
  % set entry spacing
  \setlength{\itemsep}{#2\baselineskip}}}
 {\end{list}}
\usepackage{calc}
\newcommand{\CSLBlock}[1]{\hfill\break\parbox[t]{\linewidth}{\strut\ignorespaces#1\strut}}
\newcommand{\CSLLeftMargin}[1]{\parbox[t]{\csllabelwidth}{\strut#1\strut}}
\newcommand{\CSLRightInline}[1]{\parbox[t]{\linewidth - \csllabelwidth}{\strut#1\strut}}
\newcommand{\CSLIndent}[1]{\hspace{\cslhangindent}#1}
\setlength{\emergencystretch}{3em} % prevent overfull lines
\providecommand{\tightlist}{%
  \setlength{\itemsep}{0pt}\setlength{\parskip}{0pt}}
\makeatletter
\@ifpackageloaded{subfig}{}{\usepackage{subfig}}
\@ifpackageloaded{caption}{}{\usepackage{caption}}
\captionsetup[subfloat]{margin=0.5em}
\AtBeginDocument{%
\renewcommand*\figurename{Figure}
\renewcommand*\tablename{Table}
}
\AtBeginDocument{%
\renewcommand*\listfigurename{List of Figures}
\renewcommand*\listtablename{List of Tables}
}
\newcounter{pandoccrossref@subfigures@footnote@counter}
\newenvironment{pandoccrossrefsubfigures}{%
\setcounter{pandoccrossref@subfigures@footnote@counter}{0}
\begin{figure}\centering%
\gdef\global@pandoccrossref@subfigures@footnotes{}%
\DeclareRobustCommand{\footnote}[1]{\footnotemark%
\stepcounter{pandoccrossref@subfigures@footnote@counter}%
\ifx\global@pandoccrossref@subfigures@footnotes\empty%
\gdef\global@pandoccrossref@subfigures@footnotes{{##1}}%
\else%
\g@addto@macro\global@pandoccrossref@subfigures@footnotes{, {##1}}%
\fi}}%
{\end{figure}%
\addtocounter{footnote}{-\value{pandoccrossref@subfigures@footnote@counter}}
\@for\f:=\global@pandoccrossref@subfigures@footnotes\do{\stepcounter{footnote}\footnotetext{\f}}%
\gdef\global@pandoccrossref@subfigures@footnotes{}}
\@ifpackageloaded{float}{}{\usepackage{float}}
\floatstyle{ruled}
\@ifundefined{c@chapter}{\newfloat{codelisting}{h}{lop}}{\newfloat{codelisting}{h}{lop}[chapter]}
\floatname{codelisting}{Listing}
\newcommand*\listoflistings{\listof{codelisting}{List of Listings}}
\makeatother
\usepackage{bookmark}
\IfFileExists{xurl.sty}{\usepackage{xurl}}{} % add URL line breaks if available
\urlstyle{same}
\definecolor{default-linkcolor}{HTML}{A50000}
\definecolor{default-filecolor}{HTML}{A50000}
\definecolor{default-citecolor}{HTML}{4077C0}
\definecolor{default-urlcolor}{HTML}{4077C0}

\hypersetup{
  pdftitle={SALIENT: ED Machine Learning Sepsis Prediction},
  pdfauthor={Iain A Bertram,},
  hidelinks,
  breaklinks=true,
  pdfcreator={LaTeX via pandoc with the Eisvogel template}}

\title{SALIENT: ED Machine Learning Sepsis Prediction}
\author{Iain A Bertram,}
\date{11 July 2025}

\usepackage[pages=all]{background}

%
% for the background color of the title page
%
\usepackage{pagecolor}
\usepackage{afterpage}
\usepackage{tikz}

%
% break urls
%
\PassOptionsToPackage{hyphens}{url}

%
% When using babel or polyglossia with biblatex, loading csquotes is recommended
% to ensure that quoted texts are typeset according to the rules of your main language.
%
\usepackage{csquotes}

%
% captions
%
\definecolor{caption-color}{HTML}{777777}
\usepackage[font={stretch=1.2}, textfont={color=caption-color}, position=top, skip=4mm, labelfont=bf, singlelinecheck=false, justification=raggedright]{caption}
\setcapindent{0em}

%
% blockquote
%
\definecolor{blockquote-border}{RGB}{221,221,221}
\definecolor{blockquote-text}{RGB}{119,119,119}
\usepackage{mdframed}
\newmdenv[rightline=false,bottomline=false,topline=false,linewidth=3pt,linecolor=blockquote-border,skipabove=\parskip]{customblockquote}
\renewenvironment{quote}{\begin{customblockquote}\list{}{\rightmargin=0em\leftmargin=0em}%
\item\relax\color{blockquote-text}\ignorespaces}{\unskip\unskip\endlist\end{customblockquote}}

%
% Source Sans Pro as the default font family
% Source Code Pro for monospace text
%
% 'default' option sets the default
% font family to Source Sans Pro, not \sfdefault.
%
\ifnum 0\ifxetex 1\fi\ifluatex 1\fi=0 % if pdftex
    \usepackage[default]{sourcesanspro}
  \usepackage{sourcecodepro}
  \else % if not pdftex
    \usepackage[default]{sourcesanspro}
  \usepackage{sourcecodepro}

  % XeLaTeX specific adjustments for straight quotes: https://tex.stackexchange.com/a/354887
  % This issue is already fixed (see https://github.com/silkeh/latex-sourcecodepro/pull/5) but the
  % fix is still unreleased.
  % TODO: Remove this workaround when the new version of sourcecodepro is released on CTAN.
  \ifxetex
    \makeatletter
    \defaultfontfeatures[\ttfamily]
      { Numbers   = \sourcecodepro@figurestyle,
        Scale     = \SourceCodePro@scale,
        Extension = .otf }
    \setmonofont
      [ UprightFont    = *-\sourcecodepro@regstyle,
        ItalicFont     = *-\sourcecodepro@regstyle It,
        BoldFont       = *-\sourcecodepro@boldstyle,
        BoldItalicFont = *-\sourcecodepro@boldstyle It ]
      {SourceCodePro}
    \makeatother
  \fi
  \fi

%
% heading color
%
\definecolor{heading-color}{RGB}{40,40,40}
\addtokomafont{section}{\color{heading-color}}
% When using the classes report, scrreprt, book,
% scrbook or memoir, uncomment the following line.
%\addtokomafont{chapter}{\color{heading-color}}

%
% variables for title, author and date
%
\usepackage{titling}
\title{SALIENT: ED Machine Learning Sepsis Prediction}
\author{Iain A Bertram,}
\date{11 July 2025}

%
% tables
%

\definecolor{table-row-color}{HTML}{F5F5F5}
\definecolor{table-rule-color}{HTML}{999999}

%\arrayrulecolor{black!40}
\arrayrulecolor{table-rule-color}     % color of \toprule, \midrule, \bottomrule
\setlength\heavyrulewidth{0.3ex}      % thickness of \toprule, \bottomrule
\renewcommand{\arraystretch}{1.3}     % spacing (padding)


%
% remove paragraph indentation
%
\setlength{\parindent}{0pt}
\setlength{\parskip}{6pt plus 2pt minus 1pt}
\setlength{\emergencystretch}{3em}  % prevent overfull lines

%
%
% Listings
%
%


%
% header and footer
%
\usepackage[headsepline]{scrlayer-scrpage}

\newpairofpagestyles{eisvogel-header-footer}{
  \clearpairofpagestyles
  \ihead*{\hspace{1cm}}
  \chead*{OFFICIAL: Sensitive}
  \ohead*{Page \thepage}
  \ifoot*{\hspace{1cm}}
  \cfoot*{Version 0.1}
  \ofoot*{\hspace{1cm}}
  \addtokomafont{pageheadfoot}{\upshape}
}
\pagestyle{eisvogel-header-footer}


\backgroundsetup{
scale=1,
color=black,
opacity=0.5,
angle=0,
contents={%
  \includegraphics[width=\paperwidth,height=\paperheight]{headers.pdf}
  }%
}

%
% Define watermark
%

\begin{document}

\begin{titlepage}
\newgeometry{top=2cm, right=4cm, bottom=3cm, left=4cm}
\tikz[remember picture,overlay] \node[inner sep=0pt] at (current page.center){\includegraphics[width=\paperwidth,height=\paperheight]{Doc2.pdf}};
\newcommand{\colorRule}[3][black]{\textcolor[HTML]{#1}{\rule{#2}{#3}}}
\begin{flushleft}
\noindent
\\[-1em]
\color[HTML]{FFFFFF}
\makebox[0pt][l]{\colorRule[360049]{1.3\textwidth}{0pt}}
\par
\noindent

% The titlepage with a background image has other text spacing and text size
{
  \setstretch{2}
  \vfill
  \vskip -8em
  \noindent {\huge \textbf{\textsf{SALIENT: ED Machine Learning Sepsis
Prediction}}}
    \vskip 2em
  \noindent {\Large \textsf{Iain A Bertram,} \vskip 0.6em \textsf{11
July 2025}}
  \vfill
}


\end{flushleft}
\end{titlepage}
\restoregeometry
\pagenumbering{arabic}

% don't generate the default title
% \maketitle


\section{About this document}\label{about-this-document}

This technical document has been developed as part of the `Gen Med
Project' by the Commission on Excellence and Innovation in Health
(CEIH). The project forms part of a program of work that is funded
through the Acute Models of Care Grant 2022 by the Medical Research
Future Fund (MRFF). The project's primary goal is to reduce unwarranted
clinical variation in general medicine, using a data analytics and
machine learning approach.

Sepsis was identified as a focus area from early analysis that looked at
high bed day consumption and high opportunity Diagnosis Related Groups
(DRGs) that included shortness of breath as a symptom.

\section{Document Revisions}\label{document-revisions}

\begin{longtable}[]{@{}
  >{\raggedright\arraybackslash}p{(\linewidth - 6\tabcolsep) * \real{0.1739}}
  >{\raggedright\arraybackslash}p{(\linewidth - 6\tabcolsep) * \real{0.2464}}
  >{\raggedright\arraybackslash}p{(\linewidth - 6\tabcolsep) * \real{0.2899}}
  >{\raggedright\arraybackslash}p{(\linewidth - 6\tabcolsep) * \real{0.2899}}@{}}
\toprule\noalign{}
\begin{minipage}[b]{\linewidth}\raggedright
No.
\end{minipage} & \begin{minipage}[b]{\linewidth}\raggedright
Date
\end{minipage} & \begin{minipage}[b]{\linewidth}\raggedright
Description
\end{minipage} & \begin{minipage}[b]{\linewidth}\raggedright
Person
\end{minipage} \\
\midrule\noalign{}
\endhead
\bottomrule\noalign{}
\endlastfoot
0.1 & 30/10/24 & First Draft Document describing the ED Sepsis
Presentation ML model, and the various verification requirements for the
model. & Iain Bertram \\
& & & \\
& & & \\
\end{longtable}

\pagebreak

\section{Introduction}\label{introduction}

A new adult sepsis pathway has been proposed for use with public
hospitals within South Australia. This pathway is intended to identify
patients who are at risk of developing Sepsis and initiating a
standardised treatment programme. The patients are identified using
Rapid Detection and Response (RDR) Observation Charts as implemented
within the public hospital Electronic Medical Record (EMR) using the
Sunrise Deteriorating Patient Reference Guide {[}1{]}.

There are two pathways for flagging a patient as being at risk of Sepsis
and requiring either review by a Senior Medical Officer or a Medical
Emergency response. The Purple pathway is triggered if any of the
patient's observations in the Rapid Detection and Response (RDR) are in
the purple zone (Table \ref{tbl:table1}). The Red pathway is triggered
if there are two or more red zone observations.

This report describes the development of a Machine Learning (ML) model
that can be used in place of these pathways. I.e. to develop machine
learning models that identify people who will be admitted to hospital
and will have a sepsis diagnosis (based on ICD-10 codes) on inpatient
discharge based on measurement of the patient's vital signs.

\begin{longtable}[]{@{}
  >{\raggedright\arraybackslash}p{(\linewidth - 12\tabcolsep) * \real{0.3200}}
  >{\centering\arraybackslash}p{(\linewidth - 12\tabcolsep) * \real{0.1200}}
  >{\centering\arraybackslash}p{(\linewidth - 12\tabcolsep) * \real{0.0960}}
  >{\centering\arraybackslash}p{(\linewidth - 12\tabcolsep) * \real{0.1200}}
  >{\centering\arraybackslash}p{(\linewidth - 12\tabcolsep) * \real{0.1200}}
  >{\centering\arraybackslash}p{(\linewidth - 12\tabcolsep) * \real{0.1040}}
  >{\centering\arraybackslash}p{(\linewidth - 12\tabcolsep) * \real{0.1200}}@{}}
\caption{\label{tbl:table1}Rapid Detection and Response (RDR) Alert
Triggers}\tabularnewline
\toprule\noalign{}
\begin{minipage}[b]{\linewidth}\raggedright
Measure
\end{minipage} & \begin{minipage}[b]{\linewidth}\centering
Purple
\end{minipage} & \begin{minipage}[b]{\linewidth}\centering
Red
\end{minipage} & \begin{minipage}[b]{\linewidth}\centering
Yellow
\end{minipage} & \begin{minipage}[b]{\linewidth}\centering
Yellow
\end{minipage} & \begin{minipage}[b]{\linewidth}\centering
Red
\end{minipage} & \begin{minipage}[b]{\linewidth}\centering
Purple
\end{minipage} \\
\midrule\noalign{}
\endfirsthead
\toprule\noalign{}
\begin{minipage}[b]{\linewidth}\raggedright
Measure
\end{minipage} & \begin{minipage}[b]{\linewidth}\centering
Purple
\end{minipage} & \begin{minipage}[b]{\linewidth}\centering
Red
\end{minipage} & \begin{minipage}[b]{\linewidth}\centering
Yellow
\end{minipage} & \begin{minipage}[b]{\linewidth}\centering
Yellow
\end{minipage} & \begin{minipage}[b]{\linewidth}\centering
Red
\end{minipage} & \begin{minipage}[b]{\linewidth}\centering
Purple
\end{minipage} \\
\midrule\noalign{}
\endhead
\bottomrule\noalign{}
\endlastfoot
& & Low & & & High & \\
Respiration (breaths/min) & 7 & & 10 & 21 & 26 & 31 \\
O\textsubscript{2} Saturation (\%) & 88 & 91 & 94 & NA & NA & NA \\
O\textsubscript{2} Flow (L/min) & NA & NA & NA & 5 & 7 & 8 \\
Blood Pressure Systolic (mm Hg) & 89 & 99 & NA & 170 & 180 & 200 \\
Pulse Rate (beats/min) & 39 & 49 & 59 & 100 & 120 & 140 \\
Temperature (ºC) & & 35 & 35.5 & 38.1 & 38.6 & \\
Level of Consciousness & NA & NA & NA & & 2 & 3 \\
\end{longtable}

\newpage

\section{Cohort}\label{cohort}

The data for this study was sourced from the South Australian Electronic
Medical Record system (EMR). The data set is composed of Emergency
Department (ED) presentations followed by an inpatient (IP) admission
with sepsis diagnosis identified by the ICD-10 codes, Table
\ref{tbl:tableicd-10} \footnote{We have removed infant Sepsis ICD-10
  codes from the previous work.} (based on a journey, see appendix
\ref{sec:Journey}).

Specifically we only include the first ED presentation in a journey.
This excludes presentations that are part of the admission process when
the patient is transferred from one hospital to another. We also require
that at least two of the following vital signs are recorded in the EMR
during the presentation: respiration, O\textsubscript{2} Saturation,
systolic blood pressure, pulse rate and temperature\footnote{O\textsubscript{2}
  Flow and the sedation score are not included as these are not usually
  filled out if the patient is not receiving oxygen or is awake and
  aware.} (Table \ref{tbl:table1}).

The IP admission is required to be the first episode of care (EoC) after
any ED presentations \footnote{Note to self, need to look at joining
  transfers. See CAP and COPD analyses.}. IP admissions where the
patient does not leave the ED are excluded (these are identified by an
IP discharge date/time within 60 minutes of the discharge date/time of
the ED presentation and a location that includes the string
``ED-Admin''.)

The training data set includes all presentations to Emergency
Departments at metropolitan hospitals in the calendar year 2023 in which
at least two observations of vital signs of interest have been made (see
Table \ref{tbl:table1}). There are \(Y\) ED presentations with at least
two vital signs recorded which included 2,761 presentations that
resulted in an inpatient admission which includes at least one ICD-10
code matching a Sepsis diagnosis. Note, \(X\) (14\%) out of a total
\(Y\) ED presentations do not have at least two observations of vital
signs recorded in the EMR and have been excluded from the analysis.

Two verification data sets are used. The first, the Metropolitan
Verification dataset, is created from ED presentations from 2024. There
are \(Y_2\) presentations with at least two vital signs recorded of
which \(X\) \$result in an inpatient admission which includes at least
one ICD-10 code matching a Sepsis diagnosis. The second, the Country
Verification dataset, is created using Emergency presentations at
country hospitals that are using the EMR. There are 149,193
presentations from the 1 January 2023 through 31 December 2024 that have
at least two recorded vital signs which results in \(X\) inpatient
admission.

\newpage

\section{Appendices}\label{appendices}

\subsection{Patient Journey}\label{sec:Journey}

Descriptions and variable names listed below are based on the SA Health
EMR. Episodes of Care are joined into a journey if they meet the
following criteria. If the value is set to zero, the EoC is part of the
same journey.

\begin{enumerate}
\def\labelenumi{\arabic{enumi}.}
\tightlist
\item
  Check if the previous CHARTGUID entry (for the same\\
  CLIENTGUID order by discharge datetime) is the same as the current
  CHARTGUID.\\
  If yes, set the value to 0.
\item
  Check if the time difference between the previous discharge datetime
  (for the same) and current admission datetime is within 6 hours
  (either before or after).\\
  If yes, set the value to 0.
\item
  Check if the time difference between the previous discharge datetime
  (for the same CLIENTGUID) and current admission datetime is within 24
  hours (either before or after) and the
  `previous\_discharge\_disposition' (for the same CLIENTGUID order by
  admission datetime) is in this list (`IP Other hosp - Down', `IP Other
  hosp - Up', `IP Other Hospital - DOWN',`IP Other Hospital - UP').\\
  If yes, set the value to 0.
\item
  Check if the previous discharge datetime (for the same CLIENTGUID
  order by discharge datetime) is before the current admission datetime
  and the previous admission datetime (for the same CLIENTGUID order by
  admission datetime) date is after the current admission datetime.\\
  If yes, set the value to 0.
\item
  Check if the time difference between the previous discharge datetime
  (for the same CLIENTGUID order by admission datetime) and current
  admission datetime is within 24 hours (either before or after) and the
  previous episode\_of\_care(for the same CLIENTGUID order by admission
  datetime) is `Rehabilitation' and the current episode\_of\_care is
  `Hospital at Home - Rehab at Home'.\\
  If yes, set the value to 0.
\item
  Check if the time difference between the previous admission datetime
  (for the same CLIENTGUID order by admission datetime) and current
  admission datetime is within 24 hours (either before or after) and the
  previous TYPECODE (for the same CLIENTGUID order by admitdate asc) is
  `Emergency' and the current source\_of\_referral is `IP
  Casualty-Emergency' and the current TYPECODE is `Inpatient'.\\
  If yes, set the value to 0.
\end{enumerate}

If none of the above conditions are met start a new journey.

\newpage

\subsection{Sepsis ICD-10 Codes}\label{sepsis-icd-10-codes}

\begin{longtable}[]{@{}
  >{\centering\arraybackslash}p{(\linewidth - 6\tabcolsep) * \real{0.1094}}
  >{\raggedright\arraybackslash}p{(\linewidth - 6\tabcolsep) * \real{0.3906}}
  >{\centering\arraybackslash}p{(\linewidth - 6\tabcolsep) * \real{0.1094}}
  >{\raggedright\arraybackslash}p{(\linewidth - 6\tabcolsep) * \real{0.3906}}@{}}
\caption{\label{tbl:tableicd-10}Sepsis Diagnosis Codes (from APC
Reference Table\_corrected - ICD-10-AM 11th edition
2019)}\tabularnewline
\toprule\noalign{}
\begin{minipage}[b]{\linewidth}\centering
Code
\end{minipage} & \begin{minipage}[b]{\linewidth}\raggedright
Diagnosis
\end{minipage} & \begin{minipage}[b]{\linewidth}\centering
Code
\end{minipage} & \begin{minipage}[b]{\linewidth}\raggedright
Diagnosis
\end{minipage} \\
\midrule\noalign{}
\endfirsthead
\toprule\noalign{}
\begin{minipage}[b]{\linewidth}\centering
Code
\end{minipage} & \begin{minipage}[b]{\linewidth}\raggedright
Diagnosis
\end{minipage} & \begin{minipage}[b]{\linewidth}\centering
Code
\end{minipage} & \begin{minipage}[b]{\linewidth}\raggedright
Diagnosis
\end{minipage} \\
\midrule\noalign{}
\endhead
\bottomrule\noalign{}
\endlastfoot
A021 & Salmonella sepsis & A412 & Sepsis due to unsp staphylococcus \\
A227 & Anthrax sepsis & A413 & Sepsis dt Haemophilus influenzae \\
A267 & Erysipelothrix sepsis & A414 & Sepsis due to anaerobes \\
A327 & Listerial sepsis & A415 & Sepsis dt oth \& unsp gram neg
organisms \\
A40 & Streptococcal sepsis & A4150 & Sepsis dt unsp Gram neg
organisms \\
A400 & Sepsis dt streptococcus group A & A4151 & Sepsis dt Escherichia
coli {[}E coli{]} \\
A401 & Sepsis dt streptococcus group B & A4152 & Sepsis due to
Pseudomonas \\
A402 & Sepsis dt streptococcus grp D \& enteroc & A4158 & Sepsis dt
other gram neg organisms \\
A403 & Sepsis dt Streptococcus pneumoniae & A418 & Other specified
sepsis \\
A408 & Other streptococcal sepsis & A419 & Sepsis, unspecified \\
A409 & Streptococcal sepsis unspecified & A427 & Actinomycotic sepsis \\
A41 & Other sepsis & B377 & Candidal sepsis \\
A410 & Sepsis due to Staphylococcus aureus & O85 & Puerperal sepsis \\
A411 & Sepsis dt other spec staphylococcus & & \\
\end{longtable}

\section{SQL Commands}\label{sql-commands}

\subsection{ED Presentations}\label{ed-presentations}

\begin{lstlisting}[language=SQL]
create or replace view DEV_DAP_CAE05_DB.SEPSIS.VW_ED_VISIT()
as SELECT * FROM (
SELECT ROW_NUMBER() OVER (PARTITION BY JOURNEY_ID ORDER BY ADMITDTM) ED_SEQ_BY_JOURNEY,
* FROM (
SELECT * FROM DEV_DAP_CAE05_DB.JNY.TB_PATIENT_JOURNEY_MAPPING_NEW  AS VISIT
WHERE visit.HOSPITAL IN (SELECT * FROM DEV_DAP_CAE05_DB.SEPSIS.VW_CONFIG_HOSPITAL)
AND TYPECODE = 'Emergency'
AND AGEONADMIT > 15
AND VISIT.ADMITDTM between '2023-01-01'  and DATEADD(month,-1,to_date(getdate()))))
WHERE ED_SEQ_BY_JOURNEY=1
\end{lstlisting}

\newpage

\section*{References}\label{references}
\addcontentsline{toc}{section}{References}

\protect\phantomsection\label{refs}
\begin{CSLReferences}{0}{1}
\bibitem[\citeproctext]{ref-RDR}
\CSLLeftMargin{1. }%
\CSLRightInline{Sunrise EMR \& PAS deteriorating patient reference guide
(July 2024). Available at:
\url{https://inside.sahealth.sa.gov.au/wps/wcm/connect/non-public+content/sa+health+intranet/it+systems/sunrise/resources/sunrise+deteriorating+patient+reference+guide}.}

\end{CSLReferences}

\end{document}
